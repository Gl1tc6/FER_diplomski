\chapter{Uvod}

U današnjem digitalnom okruženju, sigurnost informacijskih sustava postala je jedan od ključnih izazova za organizacije svih veličina. S povećanjem složenosti cyber napada i 
njihove učestalosti, tradicionalni pristup sigurnosti koji se oslanja na perimetarsku zaštitu više nije dovoljan. Organizacije moraju usvojiti proaktivne i sveobuhvatne sigurnosne 
strategije koje omogućavaju brzu detekciju, analizu i odgovor na sigurnosne prijetnje.

\textbf{Potreba za naprednim sigurnosnim rješenjima}

Moderna sigurnosna prijetnja krajolika karakterizira nekoliko ključnih trendova:

\begin{itemize}
\item \textbf{Sofisticiranost napada} - Napredni perzistentni prijetnje (APT) koriste složene tehnike za izbjegavanje tradicionalnih sigurnosnih mjera
\item \textbf{Brzina širenja} - Malware i ransomware mogu se proširiti kroz mrežu u minutama
\item \textbf{Heterogenost IT okruženja} - Organizacije koriste hibridne oblak-lokalne infrastrukture koje otežavaju sigurnosno nadziranje
\item \textbf{Nedostatak sigurnosnih stručnjaka} - Globalni nedostatak kvalificiranog osoblja otežava pravilno upravljanje sigurnošću
\end{itemize}

\textbf{Evolucija sigurnosnih tehnologija}

Kao odgovor na ove izazove, razvile su se napredne sigurnosne tehnologije koje omogućavaju:

\begin{itemize}
\item Kontinuirano nadziranje sigurnosnih događaja
\item Automatiziranu detekciju anomalija i prijetnji
\item Brz i koordiniran odgovor na incidente
\item Korelaciju podataka iz različitih izvora
\item Proaktivno `lov na prijetnje' (threat hunting)
\end{itemize}

Tri ključne kategorije ovih tehnologija su:
\begin{itemize}
\item \textbf{EDR (Endpoint Detection and Response)} - fokus na krajnje točke
\item \textbf{NDR (Network Detection and Response)} - fokus na mrežni promet
\item \textbf{XDR (Extended Detection and Response)} - integrirani pristup
\end{itemize}

\textbf{Cilj seminara}

Ovaj seminar istražuje navedene tehnologije kroz detaljnu analizu njihovih karakteristika, mogućnosti i ograničenja. Posebnu pozornost posvećujemo usporedbi vodećih komercijalnih i 
open-source rješenja, kao i preporukama za različite veličine organizacija. Cilj je pružiti sveobuhvatan pregled koji će omogućiti informirane odluke o odabiru najprikladnijih 
sigurnosnih rješenja.