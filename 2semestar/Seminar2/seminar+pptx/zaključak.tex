\chapter{Zaključak}

\textbf{Ključni nalazi istraživanja}

Analiza sigurnosnih tehnologija za detekciju i odgovor na prijetnje pokazuje jasnu evoluciju od tradicionalnih IDS/NIDS sustava prema sofisticiranim XDR platformama. Ova evolucija reflektira rastuće potrebe organizacija za proaktivnim, automatiziranim i sveobuhvatnim sigurnosnim rješenjima.

\textbf{Tehnološka evolucija}

Istraživanje pokazuje sljedeći progresivni razvoj:

\begin{enumerate}
\item \textbf{IDS/NIDS era} - Pasivna detekcija i uzbunjivanje
\item \textbf{EDR era} - Fokusirani endpoint response capabilities
\item \textbf{NDR era} - Mrežno-centrirana detection i response
\item \textbf{XDR era} - Holistička integracija svih sigurnosnih slojeva
\end{enumerate}

Matematički odnos $(NDR \cup EDR) \subset XDR$ potvrđuje da XDR predstavlja superset postojećih tehnologija, a ne njihovu zamjenu.

\textbf{Tržišne dinamike}

Analiza tržišta otkriva:
\begin{itemize}
\item \textbf{AI/ML dominaciju} - Sva vodeća rješenja integriraju napredne AI capabilities
\item \textbf{Cloud-first pristup} - Nativna cloud arhitektura postaje standard
\item \textbf{Konsolidaciju alata} - Organizacije preferiraju integrirane platforme
\item \textbf{Demokratizaciju sigurnosti} - Dostupnost naprednih capabilities i manjim organizacijama
\end{itemize}

\textbf{Komparativni uvidi}

\textbf{Performanse alata}

Na osnovu detaljne analize Snort, Suricata i Zeek sustava:
\begin{itemize}
\item \textbf{Suricata} se pokazala kao najbolji overall performer (prosjek 1.67)
\item \textbf{Zeek} excels u forenzičkoj analizi i resource efficiency
\item \textbf{Snort} predstavlja solid middle-ground opciju
\end{itemize}

\textbf{Komercijalna rješenja}

Tržišni lideri pokazuju jasnu segmentaciju:
\begin{itemize}
\item \textbf{Darktrace (19.5\%)} - Pionir u AI-driven security
\item \textbf{CrowdStrike (15.5\%)} - Premium endpoint protection leader
\item \textbf{Wazuh (13.0\%)} - Dominant open-source alternative
\item \textbf{Vectra AI (11.3\%)} - Specialized AI network detection
\end{itemize}

\textbf{Segmentacijske preporuke - prema veličini organizacije}

Istraživanje potvrđuje da ne postoji `one-size-fits-all' rješenje:

\textbf{Mala poduzeća:}
\begin{itemize}
\item Preferirane opcije: Microsoft Defender XDR, Darktrace, Wazuh
\item Ključni faktori: Jednostavnost, cijena, minimal maintenance
\item ROI focus: Quick wins i operational efficiency
\end{itemize}

\textbf{Srednja poduzeća:}
\begin{itemize}
\item Preferirane opcije: Cisco Snort, SentinelOne, Vectra AI
\item Ključni faktori: Balans performansi i cijene, skalabilnost
\item ROI focus: False positive reduction i team productivity
\end{itemize}

\textbf{Velika poduzeća:}
\begin{itemize}
\item Preferirane opcije: CrowdStrike Falcon, Cortex XDR, Vectra AI
\item Ključni faktori: Advanced capabilities, enterprise integration
\item ROI focus: Advanced threat protection i operational optimization
\end{itemize}

\textbf{Buduće perspektive}

\textbf{Tehnološki trendovi}

Očekujemo sljedeće razvojne smjerove:

\begin{itemize}
\item \textbf{Autonomous security} - Potpuno automatizirani odgovor na incidente
\item \textbf{Quantum-resistant security} - Priprema za post-quantum cryptography
\item \textbf{Zero Trust integration} - Dublja integracija s Zero Trust arhitekturama
\item \textbf{Edge security} - Proširenje na IoT i edge computing okruženja
\item \textbf{Privacy-preserving analytics} - Balans između sigurnosti i privatnosti
\end{itemize}

\textbf{Tržišne evolucije}

\begin{itemize}
\item \textbf{Platform konsolidacija} - Fewer, ali comprehensive vendors
\item \textbf{Democratization} - Advanced capabilities dostupne manjim organizacijama
\item \textbf{Specialization} - Sector-specific security solutions
\item \textbf{Outcome-based pricing} - Shift od capacity-based prema results-based pricing
\end{itemize}

\textbf{Praktične implikacije}

\textbf{Za IT profesionalce}

\begin{enumerate}
\item \textbf{Skill development} - Ulaganje u AI/ML i cloud security expertise
\item \textbf{Architecture thinking} - Holistički pristup sigurnosnoj arhitekturi
\item \textbf{Automation focus} - Automatizacija kao competitive advantage
\item \textbf{Business alignment} - Povezivanje sigurnosnih investicija s business outcomes
\end{enumerate}

\textbf{Za organizacije}

\begin{enumerate}
\item \textbf{Strategic planning} - 3-5 godina visibility u sigurnosnoj roadmap
\item \textbf{Vendor relationships} - Partnership approach s key security vendors
\item \textbf{Hybrid approaches} - Kombinacija commercial i open-source rješenja
\item \textbf{Continuous assessment} - Regular evaluation sigurnosne efikasnosti
\end{enumerate}

\textbf{Završne preporuke}

\textbf{Immediate actionables}

\begin{itemize}
\item \textbf{Assessment} - Audit postojećih sigurnosnih capabilities
\item \textbf{Gap analysis} - Identifikacija nedostataka u detection/response coverage
\item \textbf{Pilot testing} - Controlled evaluation odabranih rješenja
\item \textbf{ROI planning} - Quantified business case za sigurnosne investicije
\end{itemize}

\textbf{Long-term considerations}

\begin{itemize}
\item \textbf{Platform approach} - Migracija prema integriranim platforms umjesto point solutions
\item \textbf{Cloud readiness} - Priprema za cloud-first security architectures
\item \textbf{Skill investment} - Continuous learning i certification programs
\item \textbf{Threat landscape monitoring} - Active tracking evolving threat landscape
\end{itemize}

Sigurnosni krajolika kontinuirano evoluira, a organizacije koje će uspješno navigirati ovim promjenama bit će one koje kombiniraju tehnološku inovaciju s promišljenim strategijskim planiranjem i kontinuiranim ulaganjem u ljudski kapital. XDR tehnologije predstavljaju značajan korak naprijed, ali njihov uspjeh ovisi o proper implementation i ongoing optimization prema specifičnim organizacijskim potrebama.