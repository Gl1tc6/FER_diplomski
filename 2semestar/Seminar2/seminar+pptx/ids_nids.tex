\chapter{IDS i NIDS sustavi}

\textbf{Definicija i osnove}

IDS (Intrusion Detection System) predstavlja temeljnu sigurnosnu tehnologiju za skeniranje sustava i detekciju upada. NIDS (Network Intrusion Detection System) je specijalizirana varijanta IDS-a fokusirana na skeniranje mrežnog prometa i detekciju mrežnih upada. Ovi sustavi predstavljaju prethodnice modernih Detection and Response tehnologija.

\textbf{Ključne razlike između IDS/NIDS i DR sustava}

\textbf{Osnovna funkcionalnost}
\begin{itemize}
\item \textbf{IDS/NIDS sustavi} - Pasivno nadgledanje i uzbunjivanje o detektiranim prijetnjama
\item \textbf{Detection and Response sustavi} - Aktivno nadgledanje s mogućnostima automatskog odgovora
\end{itemize}

\textbf{Mogućnosti odgovora}
IDS sustavi su ograničeni na detekciju i uzbunjivanje, dok DR sustavi mogu imati konfigurirane automatske radnje:
\begin{itemize}
\item Automatska izolacija kompromitiranih uređaja
\item Blokiranje sumljive mrežne komunikacije
\item Pokretanje forenzičkih procesa
\item Eskalacija incidenata prema odgovornim timovima
\end{itemize}

\textbf{Vodeći IDS/NIDS alati}

\textbf{Snort}
Snort je jedan od najpoznatijih open-source IDS/IPS alata:

\textbf{Karakteristike:}
\begin{itemize}
\item IDS s mogućnostima IPS-a (Intrusion Prevention System)
\item Pravila zasnovana na prepoznavanju uzoraka (signature-based detection)
\item Lagana arhitektura pogodna za manje mreže
\item Može biti ograničen kod velikih mrežnih opterećenja
\item Jednostavan za konfiguraciju i deployment
\item Dostupnost gotovih pravila za poznate vulnerabilities
\end{itemize}

\textbf{Suricata}
Suricata predstavlja napredni IDS/IPS alat s boljim performansama:

\textbf{Karakteristike:}
\begin{itemize}
\item Paralelna obrada paketa što rezultira boljom izvedbom od Snorta
\item Podrška za multi-threading i GPU akceleraciju
\item Kompatibilnost sa Snort pravilima
\item Naprednije mogućnosti za veliku propusnost mreža
\item Built-in support za moderne mrežne protokole
\end{itemize}

\textbf{Zeek (bivši Bro)}
Zeek se razlikuje od tradicionalnih IDS alata fokusiranjem na mrežnu analizu:

\textbf{Karakteristike:}
\begin{itemize}
\item Fokus na detaljnoj analizi mrežnog prometa umjesto detekcije prijetnji putem pravila
\item Pogodan za dubinsku inspekciju mreže i zapisivanje podataka
\item Izvrsne mogućnosti za forenzičku analizu
\item Skriptni jezik za prilagođene analize
\item Strukturirani logovi za lakšu integraciju s drugim sustavima
\end{itemize}

\textbf{Tržišno pozicioniranje}

DR sustavi se tretiraju kao skuplji i bolji proizvod u odnosu na tradicionalne IDS sustave. Od DR sustava, najtraženiji su XDR-ovi zbog njihovih sveobuhvatnih mogućnosti. Ova evolucija reflektira potrebu organizacija za proaktivnijim pristupom sigurnosti koji ne samo detektira, već i automatski odgovara na prijetnje.

\textbf{Implementacijski model}

\begin{table}[h]
\centering
\begin{tabular}{|l|p{4cm}|p{4cm}|}
\hline
\textbf{Sustav} & \textbf{Primarni fokus} & \textbf{Odgovor na prijetnje} \\
\hline
IDS/NIDS & Detekcija i uzbunjivanje & Manualine intervencije potrebne \\
\hline
EDR & Krajnje točke & Automatski i poluautomatski \\
\hline
NDR & Mrežni promet & Automatski i poluautomatski \\
\hline
XDR & Holistički pristup & Orkestrirani automatski odgovor \\
\hline
\end{tabular}
\caption{Usporedba fokusa i mogućnosti odgovora različitih sustava}
\end{table}

\textbf{Hibridni pristup}

Mnoge organizacije implementiraju hibridne pristupe koji kombiniraju:
\begin{itemize}
\item Postojeće IDS/NIDS sustave za osnovnu detekciju
\item Moderna DR rješenja za naprednu analizu i odgovor
\item Centralizirane SIEM platforme za korelaciju događaja
\item SOAR sustave za orkestaciju odgovora na incidente
\end{itemize}

Ovakav pristup omogućava postupnu modernizaciju sigurnosne infrastrukture uz zadržavanje postojećih investicija.