\chapter{XDR (Extended Detection and Response)}

\textbf{Definicija i evolucija}

XDR (Extended Detection and Response) predstavlja sljedeću evoluciju sigurnosnih tehnologija koja kombinira elemente EDR i NDR sustava u jedinstvenu, integriranu platformu. XDR se često naziva "evoluiranom EDR" jer proširuje fokus s krajnjih točaka na cjelokupno IT okruženje organizacije, uključujući email, mreže, aplikacije, oblak servise i krajnje točke.

\textbf{Holistički pristup sigurnosti}

XDR rješenja nastoje riješiti fragmentaciju tradicionalnih sigurnosnih alata kroz:

\textbf{Široki spektar nadzora}
\begin{itemize}
\item \textbf{Krajnje točke} - Laptopi, desktop računala, serveri, mobilni uređaji
\item \textbf{Mrežni promet} - East-west i north-south komunikacije
\item \textbf{Email sustavi} - Detekcija phishing i malicious email prijetnji
\item \textbf{Cloud aplikacije} - SaaS i IaaS platforme
\item \textbf{Aplikacijski sloj} - Web aplikacije i API-ji
\item \textbf{Identity sustavi} - Upravljanje identitetima i pristupima
\end{itemize}

\textbf{Korelacija podataka}
XDR platforme prikupljaju i koreliraju podatke iz različitih izvora:
\begin{itemize}
\item Centralizirani pristup analizi sigurnosnih događaja
\item Automatska korelacija između različitih sigurnosnih slojeva
\item Kontekstualno povezivanje povezanih događaja
\item Smanjenje false-positive alarma kroz multi-source validaciju
\end{itemize}

\textbf{Ključne prednosti XDR rješenja}

\begin{itemize}
\item \textbf{Holističan pregled} - Potpuna vidljivost u sigurnosni status organizacije
\item \textbf{Smanjeni false-positives} - Korelacija podataka smanjuje lažne alarme
\item \textbf{Brža detekcija} - Kombiniranje različitih detection metoda
\item \textbf{Koordiniran odgovor} - Automatizirana reakcija kroz različite sustave
\item \textbf{Pojednostavljena administracija} - Jedna platforma umjesto više alata
\item \textbf{Poboljšana threat hunting} - Mogućnost praćenja prijetnji kroz različite slojeve
\end{itemize}

\textbf{Arhitekturni pristup}

\textbf{Native XDR}
Proizvodi razvijeni iz temelja kao XDR rješenja:
\begin{itemize}
\item Jedinstvena arhitektura za sve komponente
\item Optimizirana integracija između modula
\item Konzistentno korisničko iskustvo
\item Primjer: Microsoft Defender XDR
\end{itemize}

\textbf{Open XDR}
Platforme koje integriraju postojeće sigurnosne alate:
\begin{itemize}
\item Fleksibilnost u odabiru najboljih alata za svaku komponentu
\item Mogućnost zadržavanja postojećih investicija
\item Kompleksnija implementacija i održavanje
\item Primjer: Palo Alto Cortex XDR
\end{itemize}

\textbf{Matematička reprezentacija odnosa}

Odnos između različitih detection and response tehnologija može se prikazati kao:

\begin{equation}
(NDR \cup EDR) \subset XDR
\end{equation}

\begin{equation}
NIDS \subset IDS
\end{equation}

\begin{equation}
IDS \subset XDR
\end{equation}

Gdje su:
\begin{itemize}
\item $\cup$ - unija (union) skupova
\item $\subset$ - podskup (subset) relacija
\item $NDR$ - Network Detection and Response
\item $EDR$ - Endpoint Detection and Response  
\item $XDR$ - Extended Detection and Response
\item $IDS$ - Intrusion Detection System
\item $NIDS$ - Network Intrusion Detection System
\end{itemize}
Ova matematička notacija pokazuje da XDR obuhvaća funkcionalnosti NDR i EDR sustava, dok tradicionalni IDS sustavi predstavljaju podskup XDR mogućnosti.

\textbf{Primjeri vodećih XDR rješenja}

\begin{table}[h]
\centering
\begin{tabular}{|l|p{3cm}|p{5cm}|}
\hline
\textbf{Proizvod} & \textbf{Tržišni udio} & \textbf{Ključne karakteristike} \\
\hline
Microsoft Defender XDR & 6.9\% & Native integracija s Microsoft ekosustavom, AI-powered analytics \\
\hline
Palo Alto Cortex XDR & 5.6\% & Open XDR platforma, robusna integracija podataka \\
\hline
CrowdStrike XDR & - & Cloud-native arhitektura, proširenje Falcon platforme \\
\hline
Trend Micro Vision One & - & Comprehensive threat visibility, risk prioritization \\
\hline
SentinelOne Singularity XDR & - & Autonomous response, Purple AI platform \\
\hline
\end{tabular}
\caption{Vodeći XDR proizvodi prema tržišnom udjelu}
\end{table}

\textbf{Implementacijske preporuke}

\textbf{Planiranje implementacije}
\begin{enumerate}
\item \textbf{Assessment postojeće infrastrukture} - Inventar postojećih sigurnosnih alata
\item \textbf{Gap analiza} - Identifikacija nedostataka u pokrivanju
\item \textbf{Integration planning} - Strategija integracije s postojećim sustavima
\item \textbf{Pilot implementacija} - Postupno proširivanje kroz organizaciju
\end{enumerate}

\textbf{Ključni faktori uspjeha}
\begin{itemize}
\item \textbf{Executive podrška} - Potrebna podrška top managementa
\item \textbf{Cross-team suradnja} - Koordinacija između IT i sigurnosnih timova
\item \textbf{Skill development} - Edukacija osoblja za korištenje XDR platformi
\item \textbf{Process optimization} - Prilagodba postojećih sigurnosnih procesa
\end{itemize}

\textbf{Budućnost XDR tehnologija}

XDR tehnologije kontinuirano evoluiraju prema:
\begin{itemize}
\item Većoj automatizaciji odgovora na incidente
\item Integraciji s SOAR (Security Orchestration, Automation and Response) platformama
\item Naprednijem machine learning i AI mogućnostima
\item Boljoj integraciji s cloud-native okruženjima
\item Proširenju na IoT i OT (Operational Technology) sustave
\end{itemize}