\chapter{Komparativna analiza: Snort vs Suricata vs Zeek}

\textbf{Metodologija evaluacije}

Na osnovu istraživanja provedenog 2022. godine u radu "Evaluating the Efficacy of Network Forensic Tools: A Comparative Analysis of Snort, Suricata, and Zeek in Addressing Cyber Vulnerabilities", analizirana su tri vodeća mrežna sigurnosna alata. Testiranje je provedeno na simuliranom mrežnom okruženju s fokusom na tri ključna kriterija.

\textbf{Kriteriji procjene}

\textbf{Točnost detekcije}
Mjerena kroz:
\begin{itemize}
\item Broj ispravnih pozitivnih detekcija (true positives)
\item Broj lažno pozitivnih slučajeva (false positives)
\item Sposobnost prepoznavanja poznatih vulnerability signatures
\item Detekcija anomalnih mrežnih obrazaca
\end{itemize}

\textbf{Performanse}
Analizirane kroz:
\begin{itemize}
\item Brzina obrade mrežnog prometa
\item Latencija u analizi paketa
\item Sposobnost rada s velikim opterećenjima
\item Stabilnost sustava pod stresom
\end{itemize}

\textbf{Resursna potrošnja}
Mjerena kao:
\begin{itemize}
\item CPU utilizacija
\item Memorijska potrošnja
\item Disk I/O zahtjevi
\item Mrežna bandwidth potreba
\end{itemize}

\textbf{Rezultati evaluacije}

\begin{table}[h]
\centering
\begin{tabular}{|l|c|c|c|c|}
\hline
\textbf{Alat} & \textbf{Točnost detekcije} & \textbf{Performanse} & \textbf{Resursna potrošnja} & \textbf{Prosjek} \\
\hline
Suricata & 1 & 1 & 3 & \textbf{1.67} \\
\hline
Zeek & 3 & 2 & 1 & \textbf{2.00} \\
\hline
Snort & 2 & 3 & 2 & \textbf{2.33} \\
\hline
\end{tabular}
\caption{Rangiranje alata (1 = najbolji, 3 = najlošiji)}
\end{table}

\textbf{Detaljni rezultati po kategorijama}

\textbf{Točnost detekcije - Suricata (1. mjesto)}
\begin{itemize}
\item \textbf{Najbolji rezultati} u prepoznavanju prijetnji
\item Napredni detection engine s multi-threading podrškom
\item Efikasna korelacija između različitih mrežnih slojeva
\item Manje false-positive alarma u odnosu na konkurenciju
\end{itemize}

\textbf{Performanse - Suricata (1. mjesto)}
\begin{itemize}
\item \textbf{Najbrža obrada} zbog paralelne arhitekture
\item Multi-threading omogućava bolje iskorištenje modernih CPU-ova
\item GPU akceleracija za kompleksne pattern matching operacije
\item Optimizirana za high-throughput mrežna okruženja
\end{itemize}

\textbf{Resursna potrošnja - Zeek (1. mjesto)}
\begin{itemize}
\item \textbf{Najmanji utjecaj} na performanse sustava
\item Efikasno upravljanje memorijom
\item Optimizirana arhitektura za kontinuiran rad
\item Minimalna CPU potrošnja u idle stanju
\end{itemize}

\textbf{Specifičnosti pojedinih alata}

\textbf{Snort - Karakteristike}
\textbf{Prednosti:}
\begin{itemize}
\item Umjerena potrošnja resursa
\item Široka podrška zajednice i dokumentacija
\item Jednostavnost konfiguracije za osnovne scenarije
\item Kompatibilnost s mnogim SIEM sustavima
\end{itemize}

\textbf{Nedostaci:}
\begin{itemize}
\item Najveće kašnjenje u analizi prometa
\item Više lažno pozitivnih detekcija
\item Ograničene performanse kod velikih mreža
\item Single-threaded arhitektura
\end{itemize}

\textbf{Suricata - Karakteristike}
\textbf{Prednosti:}
\begin{itemize}
\item Najbolja kombinacija točnosti i performansi
\item Paralelna obrada omogućava skalabilnost
\item Kompatibilnost sa Snort pravilima
\item Napredni inspection capabilities
\end{itemize}

\textbf{Nedostaci:}
\begin{itemize}
\item Najveće zahtjevi za resursima
\item Složenija konfiguracija za optimalne performanse
\item Potrebna veća memorija za napredne značajke
\end{itemize}

\textbf{Zeek - Karakteristike}
\textbf{Prednosti:}
\begin{itemize}
\item Najbolje za forenzičku analizu
\item Detaljni strukturirani logovi
\item Izvrsne mogućnosti za threat hunting
\item Mogućnost custom scripting
\end{itemize}

\textbf{Nedostaci:}
\begin{itemize}
\item Nije dizajniran za real-time detekciju napada
\item Zahtijeva više stručnog znanja za konfiguraciju
\item Kompleksniji za integraciju s existing infrastructure
\end{itemize}

\textbf{Preporuke za implementaciju}

\textbf{Suricata - Optimalno za}
\begin{itemize}
\item Sustave koji trebaju brzu i točnu detekciju prijetnji
\item Organizacije s high-throughput mrežnim zahtjevima
\item Okruženja gdje su performanse kritične
\item Hybrid cloud/on-premise arhitekture
\end{itemize}

\textbf{Snort - Optimalno za}
\begin{itemize}
\item Manje mreže s ograničenim resursima
\item Organizacije koje traže stabilno i dobro podržano rješenje
\item Environments s postojećom Snort infrastrukturom
\item Budget-conscious implementacije
\end{itemize}

\textbf{Zeek - Optimalno za}
\begin{itemize}
\item Forenzičku analizu i post-incident investigation
\item Threat hunting aktivnosti
\item Research i advanced analytics
\item Okruženja gdje je dubinska analiza prioritet
\end{itemize}

\textbf{Zaključak komparativne analize}

Rezultati pokazuju da ne postoji univerzalno "najbolji" alat, već je izbor ovisan o specifičnim potrebama organizacije:

\begin{itemize}
\item \textbf{Za općenite sigurnosne potrebe}: Suricata pruža najbolji ukupni rezultat
\item \textbf{Za ograničene resurse}: Snort predstavlja dobru kompromis opciju
\item \textbf{Za naprednu analizu}: Zeek je nezamjenjiv za forenzičke potrebe
\end{itemize}

Mnoge organizacije implementiraju hibridne pristupe koristeći kombinaciju ovih alata za različite scenarije korištenja.