\chapter{Usporedba vodećih tržišnih rješenja}

\textbf{Tržišni udjeli IDPS segmenta}

Prema analizama PeerSpot platforme, tržište IDPS (Intrusion Detection and Prevention Software) karakteriziraju sljedeći udjeli:

\begin{table}[h]
\centering
\begin{tabular}{|l|c|}
\hline
\textbf{Proizvod} & \textbf{Tržišni udio} \\
\hline
Darktrace & 19.5\% \\
\hline
Vectra AI & 11.3\% \\
\hline
Palo Alto NATP & 7.4\% \\
\hline
Snort & 3.3\% \\
\hline
\end{tabular}
\caption{Tržišni udjeli u IDPS kategoriji}
\end{table}

\textbf{Detaljne usporedbe komercijalnih rješenja}

\textbf{Darktrace}
Darktrace dominira tržište s revolucionarnim AI pristupom sigurnosti.

\textbf{Ključne prednosti:}
\begin{itemize}
\item \textbf{Enterprise Immune System} - Samoučeći AI sustav koji se prilagođava mrežnom okruženju
\item \textbf{Stabilan rad} - Minimalan downtime s pouzdanim performansama
\item \textbf{Informativni alarmi} - Kontekstualni alarmi s minimalnim "šumom"
\item \textbf{Antigena funkcionalnost} - Instantni automatiziran odgovor na prijetnje
\item \textbf{Mrežno i email nadgledanje} - Posebice efikasno za ove domene
\end{itemize}

\textbf{Glavni nedostaci:}
\begin{itemize}
\item \textbf{Visoka cijena} - Model naplate se smatra problematičnim
\item \textbf{Ograničena endpoint zaštita} - Fokus više na mrežnu nego endpoint detekciju
\item \textbf{Brojni false-positives} - Zahtijeva značajno ručno konfiguriranje
\item \textbf{Slaba integracija} - Ograničena automatizacija s drugim alatima
\item \textbf{Dokumentacija} - Potrebno poboljšanje korisničke podrške
\end{itemize}

\textbf{Vectra AI}
Vectra AI se fokusira na AI-pogonjena rješenja za detekciju naprednih prijetnji.

\textbf{Ključne prednosti:}
\begin{itemize}
\item \textbf{Ocjene rizika} - Napredni scoring sustav za bolje prioritete
\item \textbf{AI detekcija} - Omogućava brže i učinkovitije reagiranje
\item \textbf{Microsoft integracija} - Poboljšava vidljivost prijetnji u Microsoft okruženju
\item \textbf{Povezivanje prijetnji} - Korelacija s uređajima za bolju analizu napada
\item \textbf{Cognito Streams} - Detaljan pregled mreže za lakšu detekciju
\end{itemize}

\textbf{Glavni nedostaci:}
\begin{itemize}
\item \textbf{SIEM ovisnost} - Zahtijeva integraciju sa SIEM za punu funkcionalnost
\item \textbf{Minimalni logovi} - Ograničene informacije proslijeđene SIEM sustavu
\item \textbf{Ograničena host vidljivost} - Nedostatna vidljivost na domaćinu
\item \textbf{Ograničena prilagodba} - Fewer customization mogućnosti
\item \textbf{False-positive tuning} - Težak proces podešavanja FP alarma
\end{itemize}

\textbf{Cisco Sourcefire SNORT}
Cisco Snort predstavlja enterprise verziju popularnog open-source alata.

\textbf{Ključne prednosti:}
\begin{itemize}
\item \textbf{Skalabilnost} - Jednostavno skaliranje za veće radne okoline
\item \textbf{Cisco integracija} - Izvrsna integracija s Cisco mrežnom opremom
\item \textbf{Tehnička podrška} - Profesionalna 24/7 podrška
\item \textbf{Cijena} - Kompetitivne cijene u odnosu na premium rješenja
\item \textbf{Filtriranje prometa} - Vrlo dobra zaštita od malware-a i web prijetnji
\item \textbf{Niska FP stopa} - Dobra točnost detekcije s malo false-positives
\end{itemize}

\textbf{Glavni nedostaci:}
\begin{itemize}
\item \textbf{Performanse} - Mogućnost poboljšanja brzine obrade
\item \textbf{Informativnost alarma} - Alarmi mogu biti informatiniji
\item \textbf{Kompleksno postavljanje} - Početna konfiguracija može biti izazovna
\item \textbf{Integracija} - Ograničena integracija s non-Cisco alatima
\end{itemize}

\textbf{Palo Alto Networks Advanced Threat Prevention}
Palo Alto NATP predstavlja integrirani pristup sigurnosti kroz next-generation vatrozide.

\textbf{Ključne prednosti:}
\begin{itemize}
\item \textbf{Robusna zaštita} - Napredna zaštita protiv zloćudnog koda
\item \textbf{Napredni vatrozid} - Najnapredniji vatrozid s intuitivnim sučeljem
\item \textbf{Upravljanje aplikacijama} - Kvalitetno upravljanje i propusnost mreže
\item \textbf{Filtar sadržaja} - Napredni content filtering i IP management
\item \textbf{Machine learning} - Poboljšano otkrivanje nepoznatih prijetnji
\end{itemize}

\textbf{Glavni nedostaci:}
\begin{itemize}
\item \textbf{Tehnička podrška} - Nedostatna razina tehničke podrške
\item \textbf{Složena implementacija} - Kompleksna početna instalacija
\item \textbf{Visoka cijena} - Skupo licenciranje i hardware troškovi
\item \textbf{ICAP podrška} - Nedostaje podrška za ICAP protokol
\end{itemize}

\textbf{XDR segment - vodeći proizvodi}

\textbf{CrowdStrike Falcon (15.5\% tržišni udio)}
\textbf{Ocjena korisnika:} 4.8/5 (1410 recenzija)

\textbf{Prednosti:}
\begin{itemize}
\item Napredna detekcija prijetnji u stvarnom vremenu
\item Nativna cloud arhitektura s fleksibilnošću implementacije
\item AI/ML tehnologija za detekciju i prevenciju
\item Lagani agent s minimalnim utjecajem na performanse
\item Napredne mogućnosti forenzike i threat hunting
\end{itemize}

\textbf{Nedostaci:}
\begin{itemize}
\item Viša cijena u odnosu na konkurenciju
\item Složenost prilagodbe upozorenja i izvještaja
\item Zahtijeva internetsku vezu za optimalnu zaštitu
\item Strma krivulja učenja za potpuno iskorištavanje
\end{itemize}

\textbf{Microsoft Defender XDR (6.9\% tržišni udio)}
\textbf{Ocjena korisnika:} 4.4/5 (1452 recenzije)

\textbf{Prednosti:}
\begin{itemize}
\item Besprijekorna integracija s Microsoft proizvodima
\item Cjenovno pristupačniji uz Microsoft 365 pretplatu
\item Automatizacija odgovora na incidente
\item Jednostavna implementacija za postojeće Microsoft korisnike
\end{itemize}

\textbf{Nedostaci:}
\begin{itemize}
\item Ograničene integracije s alatima trećih strana
\item Složen proces početnog postavljanja
\item Visoka potrošnja sistemskih resursa
\item Složena struktura licenciranja
\end{itemize}

\textbf{Wazuh (13.0\% tržišni udio)}
\textbf{Ocjena korisnika:} 7.5/10

\textbf{Prednosti:}
\begin{itemize}
\item Besplatna open-source platforma
\item Visoka prilagodljivost i fleksibilnost
\item Sveobuhvatna analiza logova
\item Podrška za različite platforme
\end{itemize}

\textbf{Nedostaci:}
\begin{itemize}
\item Zahtijeva značajnu tehničku stručnost
\item Ograničena profesionalna podrška
\item Nedostatna dokumentacija za complex troubleshooting
\item Može se boriti s velikim količinama podataka
\end{itemize}

\textbf{Trendovi i insights}

\textbf{AI i Machine Learning dominacija}
Sva vodeća rješenja integriraju napredne AI/ML capabilities:
\begin{itemize}
\item Darktrace s Enterprise Immune System konceptom
\item Vectra AI s fokusiranošću na AI-driven detection
\item CrowdStrike s cloud-native AI platformom
\item Palo Alto s machine learning za nepoznate prijetnje
\end{itemize}

\textbf{Hibridni pristupi}
Organizacije sve više kombiniraju:
\begin{itemize}
\item Komercijalna rješenja za kritične komponente
\item Open-source alate za specifične potrebe (Wazuh, Snort)
\item Cloud-native platforms za skalabilnost
\item On-premise rješenja za compliance zahtjeve
\end{itemize}