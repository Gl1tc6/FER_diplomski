\chapter{EDR (Endpoint Detection and Response)}

\textbf{Definicija i osnove}

EDR (Endpoint Detection and Response) je sigurnosna tehnologija koja kontinuirano nadzire krajnje točke - laptope, desktop računala, mobilne uređaje i ostalu računalnu opremu koja se može povezati na unutrašnji sustav organizacije. EDR rješenja predstavljaju evoluciju tradicionalnih antivirus programa, pružajući daleko naprednije mogućnosti za detekciju, analizu i odgovor na sigurnosne prijetnje.

\textbf{Ključne karakteristike EDR rješenja}

\textbf{Kontinuirani nadzor}
EDR sustavi rade 24/7, neprestano analizirajući aktivnosti na krajnjim točkama:
\begin{itemize}
\item Nadzor procesa i aplikacija
\item Praćenje mrežnih veza
\item Analiza datotečnih operacija
\item Monitoring registra sustava (Windows)
\item Praćenje korisničkih aktivnosti
\end{itemize}

\textbf{Usmjerenost na krajnje točke}
EDR rješenja tretiraju svaki uređaj kao potencijalni vektor napada, što im omogućava:
\begin{itemize}
\item Detaljnu analizu ponašanja na svakom uređaju
\item Izolaciju kompromitiranih uređaja
\item Forenzičku analizu sigurnosnih incidenata
\item Rollback funkcionalnosti za oporavak od napada
\end{itemize}

\textbf{Walled Garden pristup}
EDR rješenja rade po principu "walled garden" - fokusiraju se isključivo na krajnje točke unutar organizacije, što im omogućava duboku integraciju i detaljnu analizu, ali ograničava vidljivost na mrežne aktivnosti između uređaja.

\textbf{Prednosti EDR rješenja}

\begin{itemize}
\item \textbf{Visoka granularnost detekcije} - Mogućnost analize na razini procesa i datoteka
\item \textbf{Rad od doma} - Posebno korisno za zaštitu udaljenih radnika
\item \textbf{Forenzička analiza} - Detaljni logovi omogućavaju rekonstrukciju napada
\item \textbf{Brz odgovor} - Mogućnost trenutne izolacije kompromitiranih uređaja
\item \textbf{Vidljivost administratora} - Centralizada administracija svih krajnjih točaka
\end{itemize}

\textbf{Ograničenja EDR rješenja}

\begin{itemize}
\item \textbf{Ograničena mrežna vidljivost} - Ne pružaju uvid u mrežni promet između uređaja
\item \textbf{Ovisnost o agentima} - Zahtijevaju instalaciju softvera na svakom uređaju
\item \textbf{Ograničenost na uređaje} - Ne mogu detektirati napade koji zaobilaze krajnje točke
\item \textbf{Potrošnja resursa} - Kontinuirani nadzor može utjecati na performanse uređaja
\end{itemize}

\begin{table}[h]
\centering
\begin{tabular}{|l|p{8cm}|}
\hline
\textbf{Proizvod} & \textbf{Ključne karakteristike} \\
\hline
CrowdStrike Falcon & Nativna cloud arhitektura, AI/ML analiza, lagani agent \\
\hline
Microsoft Defender for Endpoint & Integracija s Microsoft ekosustavom, ugrađen u Windows \\
\hline
SentinelOne & Autonomni AI agent, offline mogućnosti, rollback funkcionalnost \\
\hline
Carbon Black (VMware) & Fokus na forenziku, napredna analiza ponašanja \\
\hline
Cybereason & Vizualizacija napada, proaktivno "lov na prijetnje" \\
\hline
\end{tabular}
\caption{Vodeći EDR proizvodi i njihove karakteristike}
\end{table}

\textbf{Implementacijske preporuke}

Pri implementaciji EDR rješenja, organizacije trebaju razmotriti:

\begin{enumerate}
\item \textbf{Pokrivenost uređaja} - Svi kritični uređaji moraju biti uključeni
\item \textbf{Mrežna povezanost} - Osigurati pouzdanu vezu s centralnim sustavom
\item \textbf{Obuka osoblja} - Sigurnosni tim mora razumjeti alate i procedure
\item \textbf{Integracija} - Povezivanje s postojećim SIEM i SOAR sustavima
\item \textbf{Testiranje} - Redovito testiranje detection i response procedura
\end{enumerate}