\chapter{NDR (Network Detection and Response)}

\textbf{Definicija i osnove}

NDR (Network Detection and Response) je sigurnosna tehnologija slična EDR sustavu, ali se fokusira na analizu mrežnog prometa umjesto na krajnje točke. NDR rješenja kontinuirano nadziru i analiziraju mrežne komunikacije kako bi identificirali sumnjive aktivnosti, anomalije i sigurnosne prijetnje koje se mogu proširiti kroz mrežu.

\textbf{Ključne karakteristike NDR rješenja}

\textbf{Mrežno usmjerenje}
NDR sustavi analiziraju mrežni promet na različitim razinama:
\begin{itemize}
\item Praćenje prometa na vatrozidima, ruterima i switchevima
\item Analiza east-west prometa (između internal sustava)
\item Nadzor north-south prometa (prema vanjskim mrežama)
\item Deep packet inspection (DPI) za detaljnu analizu sadržaja
\end{itemize}

\textbf{Analitičke tehnologije}
NDR rješenja koriste napredne analitičke metode:
\begin{itemize}
\item \textbf{Strojno učenje} - Za detekciju anomalija u mrežnom prometu
\item \textbf{Analiza ponašanja} - Prepoznavanje neobičnih komunikacijskih obrazaca
\item \textbf{Signature/Fingerprint analiza} - Prepoznavanje poznatih prijetnji
\item \textbf{Statistička analiza} - Otkrivanje odstupanja od normalnih vrijednosti
\end{itemize}

\textbf{Mrežno pokrivanje}
NDR sustavi pružaju "veći prostor" analize u odnosu na EDR:
\begin{itemize}
\item Vidljivost u cjelokupnu mrežnu infrastrukturu
\item Mogućnost otkrivanja lateral movement napada
\item Analiza komunikacije između sustava bez agenata
\item Detekcija skrivenih tunela i kovertnih kanala
\end{itemize}

\textbf{Prednosti NDR rješenja}

\begin{itemize}
\item \textbf{Bolja efikasnost} - Ranije primjećivanje napada kroz mrežnu analizu
\item \textbf{Neovisnost o agentima} - Ne zahtijeva instalaciju softvera na krajnje točke
\item \textbf{Sveobuhvatna vidljivost} - Analiza cjelokupnog mrežnog prometa
\item \textbf{Detekcija lateral movement-a} - Otkrivanje širenja napada kroz mrežu
\item \textbf{Forenzička analiza} - Mogućnost rekonstrukcije mrežnih aktivnosti
\end{itemize}

\textbf{Ograničenja NDR rješenja}

\begin{itemize}
\item \textbf{Ograničena granularnost} - Manje detaljna analiza na razini uređaja
\item \textbf{Enkriptirani promet} - Poteškoće s analizom šifriranog prometa
\item \textbf{Potreba za mrežnim pristupom} - Zahtijeva konfiguraciju mrežne infrastrukture
\item \textbf{Složenost implementacije} - Potrebno duboko razumijevanje mrežnih protokola
\end{itemize}

\begin{table}[h]
\centering
\begin{tabular}{|l|p{8cm}|}
\hline
\textbf{Proizvod} & \textbf{Ključne karakteristike} \\
\hline
Darktrace & AI-pogonjeno samoučenje, Enterprise Immune System \\
\hline
Vectra AI & AI detekcija, fokus na ransomware i lateral movement \\
\hline
Cisco Stealthwatch & Integracija s Cisco mrežnom opremom, NetFlow analiza \\
\hline
\end{tabular}
\caption{Vodeći NDR proizvodi i njihove karakteristike}
\end{table}

\textbf{Tehnologije korištene u NDR sustavima}

\textbf{Mrežno prikupljanje podataka}
\begin{itemize}
\item \textbf{Network TAPs} - Fizički pristup mrežnom prometu
\item \textbf{SPAN portovi} - Kopiranje prometa s mrežnih uređaja
\item \textbf{Flow podatci} - NetFlow, sFlow, IPFIX protokoli
\item \textbf{Packet capture} - Snimanje kompletnih mrežnih paketa
\end{itemize}

\textbf{Analitički motori}
\begin{itemize}
\item \textbf{Behavior analytics} - Prepoznavanje normalnih i anomalnih obrazaca
\item \textbf{Threat intelligence} - Usporedba s bazama poznatih prijetnji
\item \textbf{Machine learning modeli} - Kontinuirano učenje i poboljšanje detekcije
\item \textbf{Statistical analysis} - Statistička obrada mrežnih metrika
\end{itemize}

\textbf{Implementacijske preporuke}

\begin{enumerate}
\item \textbf{Mrežna arhitektura} - Planiranje pristupnih točaka za analizu prometa
\item \textbf{Bandwidth planning} - Osiguravanje dovoljne propusnosti za analizu
\item \textbf{Storage requirements} - Planiranje kapaciteta za čuvanje mrežnih podataka
\item \textbf{Integration} - Povezivanje s postojećim SIEM i SOC procesima
\item \textbf{Tuning} - Konfiguriranje za smanjenje false positive alarma
\end{enumerate}