\documentclass[12pt,a4paper]{article}
\usepackage[utf8]{inputenc}
\usepackage[croatian]{babel}
\usepackage[T1]{fontenc}
\usepackage{geometry}
\geometry{margin=2.5cm}
\usepackage{setspace}
\onehalfspacing{}
\usepackage{url}

\title{Svjesnost konteksta u prevenciji prirodnih nepogoda: \\ Analiza ALERTCalifornia sustava kao primjera sveprisutnog računarstva}
\author{Ante Čavar \\ \textit{Fakultet elektrotehnike i računarstva, Sveučilište u Zagrebu}}
\date{\textit{Zagreb, lipanj 2024.}}
\begin{document}

\maketitle

\indent U današnje vrijeme, prirodne nepogode predstavljaju značajan izazov za društvo, a njihova učestalost i intenzitet rastu zbog klimatskih promjena. 
Ovaj esej istražuje kako sustavi poput ALERTCalifornia koriste sveprisutno računarstvo za poboljšanje svjesnosti konteksta i prevenciju prirodnih nepogoda. 
Sustav ALERTCalifornia, koji se oslanja na mrežu kamera i senzora, omogućuje prikupljanje i analizu podataka u stvarnom vremenu, čime se poboljšava sposobnost 
predviđanja i odgovora na prirodne nepogode.

\indent Za početak važno je razumjeti što uopće znači kontekst tj.\ kako ga definiramo u ovom radu. Pri samoj definiciji citirati ćemo definiciju konteksta iz 
rada \textit{Towards a Better Undersanding of Context and Context-Awareness} autora A. K. Dey i G. D. Abowd, gdje kontekst definiraju kao:

\begin{quote}
\textit{Kontekst je bilo koja informacija koja nam služi za karakterizaciju situacije entiteta. Entitet je osoba, mjesto ili objekt kojeg smatramo relevantnim za 
interakciju između korisnika i aplikacije, uključujući korisnika i aplikaciju.}\cite{dey2001context}
\end{quote}

U našem slučaju entiteti su senzori i kamere koje čine sustav ALERTCalifornia, a kontekst se odnosi na informacije prikupljene tim uređajima koje pomažu u
predviđanju i odgovoru na prirodne nepogode.\cite{alertcalifornia2024} Sada moramo razumjeti još jednan važan pojam vezan uz kontekst, a to je svjesnost konteksta.
Svjesnost konteksta je karakteristika sustava koji može razumjeti i koristiti informacije iz okoline i nekakvo prethodno znanje za donošenje odluka. Aplikacije koje
su svjesne konteksta prate tko, što, gdje i kada kako bi mogle utvrditi zašto. U našem slučaju sustav ALERTCalifornia koristi senzore i kamere za prikupljanje podataka 
o vremenskim uvjetima, topografiji i drugim relevantnim informacijama kako bi stvorio sveobuhvatan pregled situacije. Ova svjesnost konteksta omogućuje sustavu da brzo 
reagira na promjene u okolini i pruži korisnicima pravovremene informacije o potencijalnim prijetnjama.

\indent Ja bi volio reći kako sam baš ja taj koji se dosjetio važnosti svjesnosti konteksta u sustavu ALERTCalifornia, no to bi bila laž. Sustav je od početka bio dizajniran
sa veoma konkretnom primjenom svjesnosti konteksta zbog svoje svrhe, a to je prevencija prirodnih nepogoda. Sustav koristi napredne algoritme za analizu podataka
prikupljenih sa senzora i kamera, što omogućuje identifikaciju potencijalnih prijetnji u stvarnom vremenu upravo pomoću stvorenog konteksta.
Takvu razinu svjesnosti konteksta \textit{izuzetno} je poželjno imati u sustavima koji nadziru kritične sustave poput šuma, rijeka te u novije vrijeme i recimo autonomna vozila.
Svaki takav sustav oslanja se na ogromnoj količini podataka prikupljenih sa `osjetila' koja onda šalju te podatke u sustav za analizu tj.\ u centralu.
Zbog svih prikupljenih informacija sustav može stvoriti kontekst entiteta, u ovom slučaju prirodnih nepogoda, i na temelju tog konteksta donositi odluke ili barem
prezentirati korisnicima informacije koje su relevantne za njihovu situaciju.

\indent Za kraj bih volio dodati kako u današnje vrijeme ne možemo više raditi `glupe' sustave koji se u potpunosti oslanjaju na operatera te gdje nismo u potpunosti
ni sigurni ako su sve informacija iz sustava relevantne a i da jesu mi ljudi se moramo namučiti sa njihovom interpretacijom. Svjesnost konteksta je ključna komponenta
za razvoj pametnih sustava koji mogu učinkovito reagirati na promjene u okolini i pružiti korisnicima pravovremene i relevantne informacije. Sustav ALERTCalifornia je odličan
primjer kako se sveprisutno računarstvo može koristiti za poboljšanje svjesnosti konteksta i prevenciju prirodnih nepogoda.
Smatram da smo tek na početku ovakvih sustava ali također vjerujem da su sustavi poput ALERTCalifornia odlična odskočna daska za daljni razvoj i napredak.

\newpage
\bibliographystyle{plain}
\bibliography{literatura}
\end{document}