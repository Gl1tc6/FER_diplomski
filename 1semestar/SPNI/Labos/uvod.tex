\chapter{Uvod}
\par Sigurnost web aplikacija je bitan dio razvoja moderne programske potpore. 
Kako se napadi na web aplikacije povećavaju iz dana u dan, potrebno je osim razumijevanja 
prijetnji i ranjivosti, imati i alate kojima ranjivosti a s njima i prijetnje možemo svesti na prihvatljivi minimum.

% \par Dva najraširenija takva alata trenutno dostupna javnosti su OWASP ZAP i Burp 
% Suite. OWASP ZAP (još znan i kao Zed Attack Proxy) je program otvorenog koda namijenjen
% za skeniranje web aplikacija tj.\ stranica. Produkt je organizacije \textit{Open Web Application 
% Security Project} (OWASP) te održavan kako od njih tako i od nemale skupine programera 
% i sigurnosnih stručnjaka iz cijelog svijeta. Alat pruža široki skup mogućnosti 
% za detekciju i testiranje ranjivosti u web aplikacijama. 

% \par Uz ZAP postoji Burp Suite, stariji i komercijalni alat namijenjen za sigurnosno testiranje 
% web aplikacija. Burp suite je razvila kompanija PortSwigger koja se također brine za njegovo 
% održavanje. Burp suite omogućuje pristup velikom rasponu alata, od skenera za traženje ranjivosti 
% pa sve do naprednih alata za \textit{fuzzing} i analizu web aplikacija. Najpoznatiji 
% alat među njima je \textit{proxy} koji omogućava korisnicima da presretnu HTTP/S zahtjeve 
% te izmijene HTTP/S zahtjeve i odgovore.

% \par OWASP ZAP i Burp Suite igraju važnu ulogu u području sigurnosti web aplikacija. 
% Prethodno navedeni i slični alati pomažu razvojnim inženjerima i sigurnosnim specijalistima u identifikaciji ranjivosti,
% razumijevanju površine napada (engl. \textit{attack surface}), vektora napada (engl. \textit{attack vector}) te u
% implementaciji efektivnih sigurnosnih rješenja.

% \par Cilj ovog rada je istražiti alat OWASP ZAP kao potencijalnu zamjenu za Burp Suite Professional testirajući
% mogućnosti oba alata u podjednakim uvjetima.
U ovom radu će se usporediti dva najraširenija alata za skeniranje i testiranje sigurnosti web aplikacija: OWASP ZAP i Burp Suite. 
Cilj je identificirati razlike u funkcionalnostima između ova dva alata, istražiti kako se mogu nadomjestiti funkcionalnosti koje ZAP 
nema, te opisati kako se pišu dodaci i proširuju mogućnosti ZAP-a. Na kraju, testirat će se nekoliko laboratorijskih vježbi koje se 
inače rješavaju pomoću alata Burp suite, koristeći ZAP.

Burp Suite je moćan, ali komercijalan alat koji zahtijeva skupu licencu za pristup svim funkcionalnostima. 
Za potrebe istraživanja i učenja, ovaj trošak može biti značajan. S druge strane, OWASP ZAP je alat otvorenog koda koji je besplatan 
za korištenje, ali potencijalno slabijih mogućnosti. Usporedbom ova dva alata, može se procijeniti koliko ZAP može biti održiva 
zamjena za Burp Suite u raznim scenarijima.

Najprije će se opisati osnovne funkcionalnosti oba alata. Zatim će se provesti nekoliko testova na različitim sigurnosnim scenarijima 
koristeći oba alata. Nakon toga, analizirat će se rezultati testova i usporediti performanse i mogućnosti ZAP-a i Burp Suite-a. 
Konačno, istražit će se mogućnosti proširenja funkcionalnosti ZAP-a kroz dodatke i skripte te predložiti rješenja za funkcionalnosti 
koje Burp Suite ima, a ZAP nema i pritom objasniti koji alat odabrati i zašto.