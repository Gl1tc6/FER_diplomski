\chapter{Zaključak}
Ovaj rad dokazuje da OWASP ZAP može poslužiti kao adekvatna zamjena za Burp Suite Professional u mnogim aspektima 
testiranja sigurnosti web aplikacija. Provedena testiranja na konkretnim laboratorijskim vježbama pokazala su da oba alata mogu 
detektirati i iskoristiti iste ranjivosti, iako ponekad različitim pristupima.
ZAP se istaknuo svojim intuitivnim sučeljem, posebice HUD-om koji olakšava interaktivno testiranje. 
Također, njegova otvorenost za proširenja i mogućnost automatizacije putem skripti čine ga izuzetno fleksibilnim alatom. 
S druge strane, Burp Suite Professional nudi nešto sofisticiraniji skup alata i bolje dokumentiranu podršku, što ga čini preferiranim 
izborom za veće organizacije i profesionalne penetracijske testere.

Ipak, postoje određene funkcionalnosti u kojima ZAP zaostaje, poput nemogućnosti skeniranja HTTP headera, što je bilo vidljivo u vježbi 
s SQL injekcijom. No, takvi nedostaci često se mogu nadomjestiti korištenjem dodataka ili alternativnih metoda.
Ključna prednost ZAP-a je svakako njegova dostupnost kao besplatnog alata otvorenog koda, što ga čini pristupačnim pojedincima i manjim 
timovima koji si ne mogu priuštiti skupe licence. Uz to, aktivna zajednica koja stoji iza ZAP-a kontinuirano radi na poboljšanjima i  
proširenjima, čime se ovaj alat neprestano unapređuje.

Zaključno, iako Burp Suite Professional i dalje drži vodeću poziciju na tržištu, OWASP ZAP se pokazao kao snažan i sposoban alat koji 
može zadovoljiti većinu potreba u području testiranja sigurnosti web aplikacija. Izbor između ova dva alata često će ovisiti o 
specifičnim potrebama projekta, raspoloživom budžetu i stručnosti tima. U svakom slučaju, postojanje kvalitetne besplatne alternative 
poput ZAP-a značajno decentralizira tržište alatima za kibernetičku sigurnost, što u konačnici doprinosi sigurnijem web okruženju te 
osigurava nadmetanje i konstantnu inovaciju.