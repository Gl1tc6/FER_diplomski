\chapter{Usporedba i nadomještanje funkcionalnosti}
%\addcontentsline{toc}{chapter}{Usporedba i nadomještanje funkcionalnosti}
Oba alata imaju slične funkcionalnosti i mogu postići iste rezultate s minimalno različitim pristupom. 
ZAP je očito namijenjen za male tvrtke i pojedince, dok je Burp Suite komercijalan alat namijenjen za velike tvrtke i korporacije. 
Iako i velike tvrtke mogu koristiti ZAP, njihovi sigurnosni specijalisti moraju proći više obuke.

Razlike u korištenju su primijećene prilikom korištenja oba alata. 
Kada se pojavi problem s Burpom ili je potrebna informacija o njihovim alatima, uvijek se može osloniti na dobro napisanu dokumentaciju. 
\textit{PortSwigger} (tvrtka koja je izgradila Burp Suite) osigurava profesionalne priručnike i video upute za korištenje svojih alata.

Kada se pojavi problem sa ZAP-om, uvelike se oslanja na rješenja koja su već dokumentirali članovi zajednice. 
Iako ZAP ima svoje priručnike i dokumentaciju, oni nisu na razini jasnoće, pokrivenosti i razumljivosti kao Burpova dokumentacija. 
Velika prednost ZAP-a je njegov HUD, koji je bio izuzetno koristan prilikom testiranja. 
U prednost ZAP-u također idu i dodaci, tj. ekstenzije, kojih ima puno više nego za Burp. 
Te ekstenzije mogu se pisati u \textit{Javi, Pythonu, Rubyu, JavaScriptu}, itd. 
Osim toga, ZAP je moguće uključiti u skripte te na taj način automatizirati testiranje.

Primjer takve skripte:

\captionsetup[figure]{labelformat=empty, labelsep=none}

\begin{figure}[H]
    \begin{verbatim}
    import time
    from zapv2 import ZAPv2

    def zap_scan(target_url, api_key='vas_api_kljuc', log='zap_rezultati.txt'):
        zap = ZAPv2(apikey=api_key)
        zap.urlopen(target_url)
        time.sleep(2)
        zap.spider.scan(target_url)
        while int(zap.spider.status('')) < 100:
            time.sleep(2)
        zap.ascan.scan(target_url)
        while int(zap.ascan.status('')) < 100:
            time.sleep(5)
        alerts = zap.core.alerts() # rjecnik alarma
        with open(log, 'w') as f:
            for alert in alerts:
                f.write(f"Alert: {alert['alert']}\nRisk: {alert['risk']}\n \
                URL: {alert['url']}\n Description: {alert['description']}\n \
                Solution: {alert['solution']}\nReference: {alert['reference']}\n)

    if __name__ == "__main__":
        target_url = input("Unesite URL zrtve: ")
        zap_scan(target_url)
    \end{verbatim}
    \captionsetup{labelformat=empty} % To remove the default "Figure" label
    \caption{\textbf{Ispis 3.18.} Jednostavna skripta za automatizaciju skeniranja u Pythonu\cite{zap_script}}
    \label{inp:script}
\end{figure}

Jedino što je potrebno napraviti je instalirati ZAP i Python, postaviti api ključ za korištenje, pokrenuti ZAP u \textit{daemon} načinu rada
te onda sa Pythonovim menadžerom za pakete instalirati paket za korištenje ZAP daemona.

\noindent
\texttt{./zap.sh -daemon -config api.key=your\_api\_key}\newline
\texttt{pip install python-owasp-zap-v2.4}